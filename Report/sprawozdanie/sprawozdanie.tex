\documentclass{classrep}
\usepackage[utf8]{inputenc}
\usepackage{color}
\usepackage{enumitem}
\usepackage{graphicx}

\studycycle{Informatyka, studia dzienne, I st.}
\coursesemester{VI}

\coursename{Komputerowe systemy rozpoznawania}
\courseyear{2019/2020}

\courseteacher{dr hab. inż. Adam Niewiadomski prof. uczelni}
\coursegroup{poniedziałek, 12:15}

\author{
\studentinfo{Mateusz Walczak}{216911} \and
\studentinfo{Konrad Kajszczak}{216790}
}

\title{Zadanie 1: Ekstrakcja cech, miary podobieństwa, klasyfikacja}
\svnurl{https://github.com/Walducha1908/KSR1}

\begin{document}
\maketitle

\section{Cel}
{Celem zadania było stworzenie aplikacji służącej do klasyfikacji artykułów prasowych metodą k-NN. Korzystając z różnych metod
wyboru słów kluczowych i ekstrakcji wektorów cech oraz istniejących miar podobieństwa, należało porównać przypisane przez naszą aplikacje kategorie artykułów do tych faktycznych. Należało również podjąć próbe opracowania własnej miary podobieństwa i/lub metryki.}

\section{Wprowadzenie}
Algorytm k najbliższych sąsiadów jest bardzo prostym klasyfikatorem probabilistycznym. Niekiedy mówi się, że algorytm k-NN jest naiwny lub leniwy. Wynika to z faktu, że nie tworzy on wewnętrznej reprezentacji danych treningowych (uczących), ale ropoczyna poszukiwanie rozwiązania dopiero podczas analizy konkretnego wzorca ze zbioru testowego. \newline

Algorytm przechowuje zbiór wszystkich wzorców uczących, względem których obliczana jest odległość wzorca testowego, zdefiniowana poprzez odpowiednią metrykę. Następnie algorytm wybiera k wzorców treningowych, nazywanych sąsiadami, do których aktualnie badany wzorzec testowy ma najmniejszą odległość. Ostateczny rezultat - kategoria, do której zostanie przypisany analizowany wzorzec - stanowi najczęściej występująca kategoria wśród k najbliższych sąsiadów.

\subsection{Metryki}

Do obliczenia odległości pomiędzy tekstami posłużyliśmy się następującymi metrykami:

\begin{itemize}[label=$\bullet$\scshape\bfseries]

\item Metryka Euklidesowa - w celu obliczenia odległości $ d_{e}(x,y) $ między dwoma punktami $ x, y $ należy obliczyć pierwiastek kwadratowy z sumy kwadratów różnic wartości współrzędnych o tych samych indeksach, zgodnie ze wzorem:
\begin{equation}
d_{e}(x,y)= \sqrt{ (y_{1} - x_{1})^2 + \cdots + (y_{n} - x_{n})^2 }
\end{equation}

\item Metryka uliczna (Manhattan, miejska) - w celu obliczenia odległości $ d_{e}(x,y) $ między dwoma punktami $ x, y $ należy obliczyć sumę wartości bezwzględnych różnic współrzędnych punktów $ x $ oraz $ y $, zgodnie ze wzorem:
\begin{equation}
d_{m}(x,y)= \sum_{k=1}^{n} | x_{k} - y_{k} |
\end{equation}

\item Metryka Czebyszewa - w celu obliczenia odległości $ d_{e}(x,y) $ między dwoma punktami $ x, y $ należy obliczyć maksymalną wartość bezwzględnych różnic współrzędnych punktów $ x $ oraz $ y $, zgodnie ze wzorem:
\begin{equation}
d_{ch}(x,y)= \max_{i} |x_{i} - y_{i}|
\end{equation}

\end{itemize}

\subsection{Wyznaczanie słów kluczowych}

Aby wyznaczyć słowa kluczowe posługujemy się poniższą metodą:

\begin{itemize}[label=$\bullet$\scshape\bfseries]
\item Term frequency - metoda polegająca na zliczeniu liczby wystąpień danego słowa we wszystkich dokumentach.
\end{itemize}

Przeprowadzamy obliczenia na zbiorze wszystkich posiadanych danych (w naszym przypadku na wszystkich artykułach) i otrzymujemy zestaw par - słowo i wartość. Taki zestaw par sortujemy malejąco po wartości i wybieramy n pierwszych słów. Wybrane n słów staje się słowami kluczowymi. \newline

Taki schemat powtarzemy $l$ razy, gdzie $l$ jest liczbą kategorii na jakie klasyfikujemy. Ostatecznie otrzymujemy $l$ zestawów słów kluczowych, przy czym każdy zestaw reprezentuje inną kategorię. Otrzymane zbiory słów kluczowych oznaczamy:

\begin{equation}
            K_{1}, K_{2}, \ldots , K_{l-1}, K_{l}.
 \end{equation}	

Otrzymany zbiór słów kluczowych będziemy używać we wszystkich iteracjach programu. Słowa kluczowe będą niezmienne, a wszystkie przeprowadzone przez nas eksperymenty będą bazowały na tym samym zbiorze słów kluczowych.

\subsection{Wyznaczanie ważonych słów kluczowych}

W celach poprawienia jakości klasyfikacji wprowadzono "ważone słowa kluczowe". Tak nazwaliśmy zestaw par - słowo kluczowe i waga (wartość zmiennoprzecinkowa), z wykorzytsaniem których przeprowadziliśmy takie same eksperymenty jak z wykorzystaniem "zwykłych" słów kluczowych, opisanych w poprzednim podpunkcie. \newline

Ważone słowa kluczowe to nic innego jak obliczony wcześniej, ten sam zestaw słów, jednak ubogacony o wagę, obliczaną zgodnie z opracowanym przez nas wzorem:
\begin{equation}
            W_{i} = \left({1 - \frac{N_{W_{i} \in K_{l}}}{l - 1}}\right)^2,
 \end{equation}	

gdzie $W_{i}$ - waga $i$-tego słowa kluczowego, $l$ - liczba kategorii, $N_{W_{i} \in K_{l}}$ - liczba kategorii słów kluczowych (innych od swojej własnej), w których $i$-te słowo kluczowe występuje. \newline

Dla jasności przeanalizujemy przykład. Niech $l = 3$, a obliczone słowa kluczowe mają postać:\newline
\begin{equation}
            K_{1} = \{{"jesien", "ogon", "krowa"}\},
\end{equation}
\begin{equation}
            K_{2} =\{{"wiosna", "ogon", "pies"}\},
\end{equation}	
\begin{equation}
            K_{3} = \{{"lato", "ogon", "krowa"}\},
\end{equation}	

Obliczmy wartości wag dla wybranych słów kluczowych z powyższego zestawu. Dla słowa "jesien" otrzymamy następującą wartość:
\begin{equation}
            W_{jesien} = \left({1 - \frac{0}{2}}\right)^2 = 1,
\end{equation}
słowo "jesien" wystapilo tylko w jednej, "swojej" kategorii, ma zatem najwiekszą możliwą wagę.

Dla słowa "krowa":
\begin{equation}
            W_{krowa} = \left({1 - \frac{1}{2}}\right)^2 = 0.25,
\end{equation}
słowo "krowa" wystąpiło w jednej dodatkowej kategorii (łącznie w dwóch).

Dla słowa "ogon":
\begin{equation}
            W_{ogon} = \left({1 - \frac{2}{2}}\right)^2 = 0,
\end{equation}
słowo "ogon" wystąpiło we wszystkich kategoriach, dlatego też uznajemy, że nie ma dla nas żadnego znaczenia, jego waga jest równa 0. \newline

Z powyższych rozważań bardzo jasno wynika, że wagi słów kluczowych mogą osiągać wartości z przedziału $ \langle0;1\rangle $.


\subsection{Cechy poddawane ekstrakcji}

 Ekstrakcja cech charakterystycznych tekstu - w tym celu tworzymy wektor cech, który opisuje tekst (w naszym przypadku artykuł) na podstawie konkretnych, zdefiniowanych cech. Poniżej znajduje się opis wszystkich cech użytych w doświadczeniu. \newline

Przyjęto następujące oznaczenia:\\
    \quad $T_{i}$ - zbiór słów do badania,\\
    \quad $K$ - stały zbiór słów kluczowych\footnote{An example footnote.}, \\
    \quad $N_{K \in T}$ - liczba wystąpień elementów zbioru K w zbiorze T, \\
    \quad $C_{i}(T,K)$ - wartość funkcji cechy. \\


\subsubsection{Liczba wystąpień wszystkich słów kluczowych w całym artykule}
Cecha opisująca liczbę słów kluczowych, które występują w całej sekcji głównej artykułu (body).
\begin{equation}
            C_{1}(T_{1},K) = N_{K \in T_{1}},
 \end{equation}	
 gdzie $T_{1}$ - zbiór słów sekcji głównej artykułu.

\subsubsection{Liczba wystąpień wszystkich słów kluczowych w tytule artykułu}
Cecha opisująca liczbę słów kluczowych, które występują w tytule artykułu (title).
\begin{equation}
            C_{2}(T_{2},K) = N_{K \in T_{2}},
 \end{equation}	
 gdzie $T_{2}$ - zbiór słów tytułu artykułu.

\subsubsection{Liczba wystąpień wszystkich słów kluczowych w sekcji daty artykułu}

Cecha opisująca liczbę słów kluczowych, które występują w sekcji daty artykułu (dateline).
\begin{equation}
            C_{3}(T_{3},K) = N_{K \in T_{3}},
 \end{equation}	
 gdzie $T_{3}$ - zbiór słów sekcji daty artykułu.

\subsubsection{Stosunek liczby wystąpień wszystkich słów kluczowych do ogólnej liczby słów w artykule}
Cecha opisująca stosunek liczby słów kluczowych, które występują w całej sekcji głównej artykułu (body), do całkowitej liczby słów występujących w części głównej.
\begin{equation}
            C_{4}(T_{4},K) = \frac{N_{K \in T_{4}}} {|T_{4}|},
 \end{equation}	
 gdzie $T_{4}$ - zbiór słów sekcji głównej artykułu, $|T_{4}|$ - liczba elementów (słów) zbioru sekcji głównej artykułu.

\subsubsection{Liczba wystąpień wszystkich słów kluczowych w pierwszych 50 słowach artykułu}
Cecha opisująca liczbę słów kluczowych, które występują w pierwszych 50 słowach sekcji głównej artykułu. Jeśli artykuł jest krótszy niż 50 słów to bierzemy pod uwagę wszystkie występujące w nim słowa.
\begin{equation}
            C_{5}(T_{5},K) = N_{K \in T_{5}},
 \end{equation}	
 gdzie $T_{5}$ - pierwsze 50 słów sekcji głównej artykułu.

\subsubsection {Liczba wystąpień wszystkich słów kluczowych w pierwszych 10\% artykułu}
Cecha opisująca liczbę słów kluczowych, które występują w pierwszych 10\% sekcji głównej artykułu.
\begin{equation}
            C_{6}(T_{6},K) = N_{K \in T_{6}},
 \end{equation}	
 gdzie $T_{6}$ - pierwsze 10\% słów sekcji głównej artykułu.

\subsubsection{Liczba wystąpień wszystkich słów kluczowych w pierwszych 20\% artykułu}
Cecha opisująca liczbę słów kluczowych, które występują w pierwszych 20\% sekcji głównej artykułu.
\begin{equation}
            C_{7}(T_{7},K) = N_{K \in T_{7}},
 \end{equation}	
 gdzie $T_{7}$ - pierwsze 20\% słów sekcji głównej artykułu.

\subsubsection{Liczba wystąpień wszystkich słów kluczowych w pierwszych 50\% artykułu}
Cecha opisująca liczbę słów kluczowych, które występują w pierwszych 50\% sekcji głównej artykułu.
\begin{equation}
            C_{8}(T_{8},K) = N_{K \in T_{8}},
 \end{equation}	
 gdzie $T_{8}$ - pierwsze 50\% słów sekcji głównej artykułu.

\subsubsection{Liczba wystąpień wszystkich słów kluczowych w pierwszym paragrafie}
Cecha opisująca liczbę słów kluczowych, które występują w pierwszym paragrafie sekcji głównej artykułu.
\begin{equation}
            C_{9}(T_{9},K) = N_{K \in T_{9}},
 \end{equation}	
 gdzie $T_{9}$ - pierwszy paragraf sekcji głównej artykułu.

\subsubsection{Liczba wystąpień wszystkich słów kluczowych w ostatnich 50 słowach artykułu}
Cecha opisująca liczbę słów kluczowych, które występują w ostatnich 50 słowach sekcji głównej artykułu. Jeśli artykuł jest krótszy niż 50 słów to bierzemy pod uwagę wszystkie występujące w nim słowa.
\begin{equation}
            C_{10}(T_{10},K) = N_{K \in T_{10}},
 \end{equation}	
 gdzie $T_{10}$ - ostatnie 50 słów sekcji głównej artykułu.

\subsubsection{Liczba wystąpień wszystkich słów kluczowych w ostatnich 10\% artykułu}
Cecha opisująca liczbę słów kluczowych, które występują w ostatnich 10\% sekcji głównej artykułu.
\begin{equation}
            C_{11}(T_{11},K) = N_{K \in T_{11}},
 \end{equation}	
 gdzie $T_{11}$ - ostatnie 10\% słów sekcji głównej artykułu.

\subsubsection{Liczba wystąpień wszystkich słów kluczowych w ostatnim paragrafie}
Cecha opisująca liczbę słów kluczowych, które występują w ostatnim paragrafie sekcji głównej artykułu.
\begin{equation}
            C_{12}(T_{12},K) = N_{K \in T_{12}},
 \end{equation}	
 gdzie $T_{12}$ - ostatni paragraf sekcji głównej artykułu.


\section{Opis implementacji}
{\color{blue}
Należy tu zamieścić krótki i zwięzły opis zaprojektowanych klas oraz powiązań
między nimi. Powinien się tu również znaleźć diagram UML  (diagram klas)
prezentujący najistotniejsze elementy stworzonej aplikacji. Należy także
podać, w jakim języku programowania została stworzona aplikacja. }

\section{Materiały i metody}
{\color{blue}
W tym miejscu należy opisać, jak przeprowadzone zostały wszystkie badania,
których wyniki i dyskusja zamieszczane są w dalszych sekcjach. Opis ten
powinien być na tyle dokładny, aby osoba czytająca go potrafiła wszystkie
przeprowadzone badania samodzielnie powtórzyć w celu zweryfikowania ich
poprawności (a zatem m.in. należy zamieścić tu opis architektury sieci,
wartości współczynników użytych w kolejnych eksperymentach, sposób
inicjalizacji wag, metodę uczenia itp. oraz informacje o danych, na których
prowadzone były badania). Przy opisie należy odwoływać się i stosować do
opisanych w sekcji drugiej wzorów i oznaczeń, a także w jasny sposób opisać
cel konkretnego testu. Najlepiej byłoby wyraźnie wyszczególnić (ponumerować)
poszczególne eksperymenty tak, aby łatwo było się do nich odwoływać dalej.}

\section{Wyniki}
{\color{blue}
W tej sekcji należy zaprezentować, dla każdego przeprowadzonego eksperymentu,
kompletny zestaw wyników w postaci tabel, wykresów itp. Powinny być one tak
ponazywane, aby było wiadomo, do czego się odnoszą. Wszystkie tabele i wykresy
należy oczywiście opisać (opisać co jest na osiach, w kolumnach itd.) stosując
się do przyjętych wcześniej oznaczeń. Nie należy tu komentować i interpretować
wyników, gdyż miejsce na to jest w kolejnej sekcji. Tu również dobrze jest
wprowadzić oznaczenia (tabel, wykresów) aby móc się do nich odwoływać
poniżej.}
\begin{figure}
	\centering
	\includegraphics[width=1\textwidth]{{Wykresy/TF_10k.png}}
	\caption{}
\end{figure}

\section{Dyskusja}
{\color{blue}
Sekcja ta powinna zawierać dokładną interpretację uzyskanych wyników
eksperymentów wraz ze szczegółowymi wnioskami z nich płynącymi. Najcenniejsze
są, rzecz jasna, wnioski o charakterze uniwersalnym, które mogą być istotne
przy innych, podobnych zadaniach. Należy również omówić i wyjaśnić wszystkie
napotakane problemy (jeśli takie były). Każdy wniosek powinien mieć poparcie
we wcześniej przeprowadzonych eksperymentach (odwołania do konkretnych
wyników). Jest to jedna z najważniejszych sekcji tego sprawozdania, gdyż
prezentuje poziom zrozumienia badanego problemu.}
\section{Wnioski}
{\color{blue}W tej, przedostatniej, sekcji należy zamieścić podsumowanie
najważniejszych wniosków z sekcji poprzedniej. Najlepiej jest je po prostu
wypunktować. Znów, tak jak poprzednio, najistotniejsze są wnioski o
charakterze uniwersalnym.}


\begin{thebibliography}{0}
\end{thebibliography}
{\color{blue} 
Na końcu należy obowiązkowo podać cytowaną w sprawozdaniu
literaturę, z której grupa korzystała w trakcie prac nad zadaniem (przykład na
końcu szablonu)}
\end{document}
